\chapter{Conclusion}\label{chap:conclusion}
This Chapter discusses the limitations and challenges imposed on this thesis, explores possible future improvements, and concludes with the main findings and remarks. 

\vspace{5mm}
{\noindent\bfseries Limitation, Challenges, \& Outlook }

This thesis faces several limiting factors. At first, the simplified ordered probit model does not incorporate the full list of explanatory variables proposed by \citet{hausman1992} that could impact the model's performance substantially, such as the S\&P 500 index that could be informative about the overall US financial market. The signed transformed volume is separately pre-transformed with natural log, while in the original study, the Box-Cox transformation is estimated jointly with maximum likelihood estimation. Furthermore, the conditional variance that accounts for conditional heteroskedasticity is simplified to only time between trade. The restrictive usage of the \verb|oglmx| package constrains how the variance could be modeled. A customized, self-developed function would be needed to replicate the ordered probit model completely. However, due to time and resource constraints, so far, this thesis only investigates IBM, a large and very liquid stock. To arrive at a robust and generalized conclusion, one should need a more representative number and size of stocks. 

Secondly, the changing nature of the high frequency financial data characteristic, compared to the 1992 period of the original paper, leaves this thesis further challenges in cleaning and pre-processing the data. One instant would be to verify whether our choice of trade direction classification's algorithm is already the most suitable, though the \% trade at a price equal to mid-quote is minor. The issue of split transaction \citep{hautsch2012} is also not addressed in this thesis, which could be a significant issue when not treated right, leading to the over-representation of the \textit{no change} price movement in the dataset.

Lastly, updating the results with the more recent market microstructure theory regarding information-based trading would be more insightful. For the time being, the relationships between order flow and price change (or between trade size and price changes) have not been discussed in depth. The thesis only revisits the test statistics and experiments from \citet{hausman1992} regarding the behavior of the market microstructure.



{\noindent\bfseries Concluding Remarks }

The ordered probit model introduced by \citet{hausman1992} provides a means to model the transaction price movement that accounts for the discreteness characteristic of price change and conditional heteroskedasticity by inducing conditional variance. This thesis attempts to reconstruct a simplified version of the ordered probit model to explore some aspects of the financial market microstructure. In particular, our model shows consistent results with the original study regarding the effect of trading sequence on the next price change. Indeed, the order flows (e.g., buy/sell/buy or sell/buy/buy) do have an impact on the next price change. Moreover, the model allows us to estimate the trade size's impact on the conditional mean of price change in different scenarios, i.e., constant price sequence of increasing price sequence. The result shows a trend of larger trade size leading to a larger price impact.

In addition, this thesis evaluates the forecasting performance of our simplified ordered probit model. The prediction ability is relatively weak. For the main study of 2023's IBM stock traded on NYSE, \% accuracy ranges only from 62-64\%. The forecasting performance is even poorer when applying for the later special periods, the Covid-19 pandemic and the financial crisis of 2008. Noticeably, it is evident that the model suffers from class imbalance, and tends to predict \textit{no change} when actually there is a price change. This issue might require us to develop a more sophisticated model for forecasting purposes. Despite the limitations and challenges, the ordered probit model provides informative findings on how stock price moves and its market microstructure behave.