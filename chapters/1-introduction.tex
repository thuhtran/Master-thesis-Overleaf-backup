\chapter{Introduction}\label{chap:introduction}

\textit{Why does one should pay attention to secondary financial markets, or namely one of those market, the stock market?} \citet{bondetal2012}, motivated to verify the impact of secondary financial market on the real economy, argue that the role of market prices' \textit{information} is critical to real decision makers (e.g. firm managers, customers, employees, etc.). There are different ways to look at this role. The first instant can be as when the decision makers make their decisions based on stock price, e.g. credit-rating. Even if one does not make decision directly, one still has the motivation to pay attention to stock price, for example when an employee's compensation is affected by the company' share price. There are also developed theoretical models that contain the "feedback effect" between the financial market and the real economy.

As stock prices are informative, and thus important, a lot of efforts have been made in the literature to investigate their behavior on both theoretical and empirical territories. Looking back through the history of the trading world, from the times of crowded trading floors to the modern days with algorithmic trading, financial data structure transforms from lower frequencies (e.g. daily, weekly, monthly, or annually) to the of (ultra-) high frequencies, such as milliseconds or nanoseconds. The high frequency financial data owns some specific characteristics, including the discreteness of transaction price and irregular time between trades \citep{tsay2013}. The discreteness of transaction price challenges the traditional assumption of continuous probability distribution. \citet{hausman1992} address this problem by introducing an ordered probit model that takes into account price discreteness. Moreover, their model also capture the effect of market microstructure elements on price, verifying "information-effect" theory by \citet{easleyohara1987}. One further application of their model, that is of the interest of many traders, is to predict the next price movement. Since then, although the so-called "low-latency" trading world continues to evolve and imposes many challenges, research directions (e.g. \citet{ohara2015}), as well as newer machine learning models are providing promising results, the ordered probit model's ability to serve both market microstructure and forecasting is worth a kind of "test-of-time" study. Hence, this thesis would revisit the \citet{hausman1992}'s ordered probit model, attempting to examine the following objectives:
\begin{itemize}
    \item \textit{Does the information of the sequence of previous price changes affect the next price change? Does the trading volume of previous trade affect the next price change?}
    \item and \textit{How effective is our model in forecasting the stock price movement?}
\end{itemize}

{\noindent {By answering these questions, we hope to learn more about how financial market microstructure works, as well as the effectiveness of the ordered probit model in our settings.}}


The remainder of the thesis is organized as follows. Chapter 2 provides an overview of the related literature, specifically in terms of predicting stock price movement. Chapter 3 describes our simplified ordered probit model and its specification. Chapter 4 discusses the empirical results of the main study. Chapter 5 presents the further applications of our model to two special periods: the Covid-19 period in 2020, and the financial crisis in 2008. Chapter 6 reviews the limitation and challenges of this thesis and concludes our findings.