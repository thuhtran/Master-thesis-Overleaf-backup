\chapter{Related Works}\label{chap:relatedwork}

In this Chapter, the related studies on stock price movement prediction are presented. The literature review first gives an overview on methods developed to forecast stock price movement, before focusing on publications that are relevant to probit/ordered probit models. Afterwards the Chapter discusses the most important characteristics of high frequency financial data, and concludes with a glance on market microstructure studies. 
\\

{\bfseries Stock Price Movement Predictability}

The quest to determine whether the stock returns are predictable or not is an open matter in the financial econometric literature. \citet{rapach_forecasting_2013} survey time-series regression models with different indicators to forecast US equity premium. These models show weak stock returns predictability. More recently, machine learning approaches are receiving attention, though their performance is still a long way to be significant. One recent study by \citet{kelly_virtue_2024} shows theoretical support for complex and large machine learning models. A comparative study by \citet{gu_empirical_2020} suggests that neural network and regression trees are the best performing models to predict asset risk premiums. 

While the aforementioned debate is on its way, another line of literature as well as the focus of this thesis is, instead, focusing on predicting the sign of returns. Being able to predict direction-of-change is economically meaningful, for instant in evaluating trading strategies. Earlier result by \citet{leichttanner1991}, considering the case of interest rate forecast, concludes that directional accuracy and profits are closely related, while that is not the case between traditional summary statistics (average absolute error and root-mean-square error) and profits. \citet{christoffersen_financial_2006} find that sign predictability does not necessarily require conditional mean predictability. \citet{christoffersenetal2007} further investigate the importance of higher-order conditional moments (skewness and kurtosis) to sign predictability. In general, we can model the direction-of-change forecasting problem as a binary classification problem (e.g. going up/down). Different variation of logistic regression models are examined, for instant by \citet{rydbergshephard2003} or by \citet{anatolyevgospodinov2010}. In the first paper, \citet{rydbergshephard2003} decompose trade-by-trade price changes into activity (move or not), direction, and size of movement. Key explanatory variables are lagged variables of the components. In a different context, \citet{anatolyevgospodinov2010}’s decomposition include the sign component and the absolute value of returns. Their set of predictors include lagged of signed of returns/absolute value, along with other macroeconomic variables (e.g. dividend-price ratio, three-month T-bill rate). Beyond the paradigm of classical binary response models, one can also find recently increasing attempts to exploit the power of machine learning (e.g. survey by \citet{bustos_stock_2020}).  

\par
{\noindent\bfseries Probit Models in Forecasting Stock Price Movement}


With an eye on using probit models, \tabref{tab:table-1} below summarizes relevant studies and their key findings. At a first glance, there are two main strands of literatures: The first strand mainly stems from the work of \citet{hausman1992}, in which they employ an ordered probit model to study transaction price movement. Some advantages of using ordered probit models are the ability to capture characteristics of high-frequency financial data such as price discreteness, as well as to study the behavior of market microstructure. Our paper thus follows the approach of \citet{hausman1992}. 

On the other hand, the second strand, including studies by \citet{nyberg2011}, \citet{nybergponka2016}, and \citet{ponka2017}, investigates different variant of binary probit models to predict the direction of monthly excess stock returns. Last but not least, the last paper by \citet{leungetal2000} compares the performance of classification models (e.g. probit/logit, probabilitistic neural network) versus level models (e.g. vector autoregression with Kalman filter, multilayered feedforward neural network) in forecasting sign of monthly returns. In general, classification models perform better than level models. 



\begin{landscape}
\begingroup
\small
\setlength{\tabcolsep}{4pt} 

\begin{longtblr}[
  caption = {Probit Models in Forecasting Stock Price Movement},
  label = {tab:table-1},
]{
  width = \linewidth,
  colspec = {Q[l,wd=0.1\linewidth] Q[l,wd=0.18\linewidth] Q[l,wd=0.12\linewidth] X[l]},
  rowhead = 1,
  lastfoot = {
  \hline
  & & &\\
},
}
\hline
\textbf{Study} & \textbf{Models} & \textbf{Data Frequency} & \textbf{Key Findings} \\
\hline
\textit{This thesis} & \textit{Ordered probit model} & \textit{Transaction (tick) level price changes }& \textit{Key Objectives: investigating the impact of sequence of trade and trade size on price, and the forecasting performance of the model.}


\\

\citet{hausman1992} & Ordered probit model & Transaction (tick) level price changes & The ordered probit model of this study captures discretness of price changes (clustering on eigths of a dollar). Sequence of past price changes and order flows (whether it is buyer-initiated or seller initiated), as well as trade size, do have an impact on transactional prices. \\ 

\citet{yangparwada2012} & Ordered probit model with GARCH(2,2) specification for residual series & Transaction (tick) level price changes & The model has an average of 71\% accuracy rate in the forecasting exercises both in- and out-of-sample. Dominant buying transactions in the past would increase the probability of price rise, while dominant selling transaction would increase the probability of the price to fall. Volume at the best bid price would have positive impact on price, and in the other way round, the volume at the best ask price have a negative impact. Conditional durations have negative effect on price, suggesting by the joint negative sign of all stocks. \\

\citet{kim2014} & Ordered probit model & Transaction (tick) level price changes & The paper revisits \citet{hausman1992} after the NYSE decimalization: For small firms, the effect of trading-related explanatory variables are more evident to the 1/16th and 1/24th range of dependent variable than 1/8th range of dependent variable. \\

\citet{nyberg2011} & Probit models: 
Static probit, Dynamic probit, Autoregressive probit \citep{kauppi_predicting_2008}, Dynamic autoregressive probit \citep{kauppi_predicting_2008}, "Error correction" dynamic autoregressive probit & Monthly excess stock returns & It seems that the direction of the excess stock return is predictable, though with low statistical power. The proposed probit models also outperform the buy-and-hold trading strategy in terms of annualized portfolio returns.
For out-of-sample results, the best performing model is the "error correction" dynamic probit model. Noticeably, models employing the recession forecast perform better than the ones incorporating the variables used in recession forecasting. \\

\citet{nybergponka2016} & Bivariate probit models & Monthly excess stock returns & The bivariate probit models study the interrelationship between US and ten industrialized countries, which allow testing the linkage of US excess return' sign forecast to other markets. In general, both in-sample and out-of-sample results show evidence that bivariate model outperform univariate model (that includes the lagged US return) in most of the markets. This suggests that the predictive power is not just limited to the lagged US return. \\ 

\citet{ponka2017} & Probit models: Static probit, Dynamic probit, Autoregressive probit, Dynamic autoregressive probit & Monthly and daily excess stock returns & The predictability of the sign of excess market returns seems to be evident in out-of-sample results. Moreover, some industries portfolios do have predictive power for market returns, some of which later also reconfirmed by robustness check with daily data. The metal and construction industry portfolios especially seem to be useful in forecasts for trading strategies. \\

\citet{leungetal2000} & \textit{Classification models}: Discriminant analysis, Logit, Probit, Probabilistic neural network;
\textit{Level models}: Adaptive exponential smoothing, Vector autoregression with Kalman filter, Multivariate transfer function, Multilayered feedforward neural network & Monthly excess stock returns & The classification models are employed to predict the sign of returns, while the level models are used to forecast the value. Generally, the classification models outperform level models in their forecasting performance i.e. have better hit rate. Noticeably, trading profits from the classification models are higher than that of the level models. \\


\end{longtblr}
\endgroup
\end{landscape}



\par

{\noindent\bfseries High Frequency Financial Data \& Market Microstructure}

