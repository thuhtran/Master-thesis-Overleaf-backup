\chapter{Theoretical Model}\label{chap:model}
\todo{chapter intro Explain the math and notation.}


{\noindent\bfseries Convention: Sign of Returns versus Price Movement}

In order to improve readability, the notion of \textit{transaction price movement} is used interchangeably with \textit{sign of transaction returns}. If the simple transaction-by-transaction, or \textbf{tick returns} is defined as:

\begin{equation}
    r_t = \frac{P_t - P_{t-1}}{P_{t-1}} ,
    \label{eq:1}
\end{equation}
where $P_t$ is the transaction price at time $t$, then the positive (negative) sign of returns is equivalent to the upward (downward) price move when \(P_t > P_{t-1}\) (\(P_t < P_{t-1}\)).

\section{Ordered Probit Model}

Our ordered probit model set-up follows \citet{hausman1992} with some simplification. 

In the context of high frequency financial data, "tick" is the smallest unit of price movement. Let $Z_k \equiv P(t_k) - P(t_{k-1})$ be an integer of observed price changes, i.e. multiples of tick. The core idea of the ordered probit model is to map the observable discrete random variable $Z_k$ with an unobserved continuous random variable $Z^*_k$. In practice, one can partition a finite number of price change categories into \textit{m} state spaces (e.g. 1-tick change, 2-tick change and so on). Depending on where $Z^*_k$ lies in the state space, $Z_k$ is determined accordingly:

\begin{equation}
Z_k =
\begin{cases}
s_1 & \text{if } Z_k^* \in A_1, \\
s_2 & \text{if } Z_k^* \in A_2, \\
\vdots & \vdots \\
s_m & \text{if } Z_k^* \in A_m,
\end{cases}
    \label{eq:2}
\end{equation}
where $s_j$ are discrete values forming state space \(\mathscr{S}\) of $Z_k$, and $A_j$ are the sets forming the partition of the state space \(\mathscr{S^*}\) of $Z^*_k$. 

The unobservable continuous random variable $Z^*_k$ follows a model such that

\begin{equation}
Z_k^* = X_k' \beta + \varepsilon_k, \quad \mathbb{E}[\varepsilon_k \mid X_k] = 0, \quad \varepsilon_k \ \text{i.n.i.d.} \ \mathcal{N}(0, \sigma_k^2),
    \label{eq:3}
\end{equation}
where \(\varepsilon_k\) is independently but not identically distributed, and $X_k$ is a \(q\times1\) vector of explanatory variables at time $t_{k-1}$.

{\noindent\bfseries Conditional Variance \(\sigma_k^2\)}

Noticeably, \citet{hausman1992}'s ordered probit model also accounts for conditional heteroskedasticity in $Z^*_k$. Intuitively, their model incorporates the clock-time effect where $Z^*_k$ can be modeled as increments of arithmetic Brownian motion as in the model of \citet{chofrees1988}. Let \(\Delta t_k\) be the time between trade $k$ and $k-1$, the variance is linear in \(\Delta t_k\):

\begin{equation}
X_k' \beta = \mu \Delta t_k, \quad \sigma_k^2 = \gamma^2 \Delta t_k.
    \label{eq:4}
\end{equation}

\citet{hausman1992} further extends the model and let the variance also depend linearly on other economic variables.
However, in the scope of this thesis, we simplify our model by adopting the first specification as in \eqref{eq:4}.

{\noindent\bfseries Explanatory Variables $X_k$ }

Our explanatory variables consist of:
\begin{itemize}
    \item $Z_{k-l}$: Three lags of price changes variable $Z_k$ (l = 1, 2, 3).
    \item $IBS_{k-l}$: Three lags of buyer/seller-initiated trade indicator (l = 1, 2, 3). The classification of trade direction is performed based on the method proposed by \citet{leeready1991}, detailed algorithm is described in Algorithm \ref{alg:IBS}. This different approach than that of \citet{hausman1992} avoids the indeterminate classification where IBS = 0.
    \begin{equation}
\mathrm{IBS}_{k-1} \;\equiv
\begin{cases}
1,  & \text{if buyer-initiated trade},\\
-1, & \text{if seller-initiated trade}.
\end{cases}
\label{eq:5}
\end{equation}
    
       \item$lnV_{k-l}IBS_{k-l}$: Three lags of signed transformed dollar volume (l = 1, 2, 3). Volume of a trade is first set to the 99.5 percentile of the traded volume distribution, if it exceeds the 99.5 percentile. Then, the traded volume is multiplied with the transaction price (in dollars), and also divided by \$100. Afterward, we take the natural log of dollar volume. 
       The interaction between transformed dollar volume and the buyer/seller-initiated indicator is taken into account to investigate the impact of trade direction on price. A positive coefficient may indicate that buyer-initiated trade would likely to move price up, and vice versa \citep{hausman1992, easleyohara1987}.
\end{itemize}

The complete specification of \(X_k' \beta\) is:
\begin{align}
X_k' \beta =\ & \beta_1 Z_{k-1} + \beta_2 Z_{k-2} + \beta_3 Z_{k-3} \nonumber \\
& + \beta_4 IBS_{k-1} + \beta_5 IBS_{k-2} + \beta_6 IBS_{k-3} \nonumber \\
& + \beta_7 \ln V_{k-1} \cdot IBS_{k-1} \nonumber \\
& + \beta_8 \ln V_{k-2} \cdot IBS_{k-2} + \beta_9 \ln V_{k-3} \cdot IBS_{k-3}, \label{eq:6}
\end{align}
with variance 

\begin{equation}
\sigma_k^2 = \gamma^2 \Delta t_k.
\label{eq:7}
\end{equation}

{\noindent\bfseries Conditional Distribution and Log-Likelihood Function}

Last but not least, under the assumption of \(\varepsilon_k \) following normal (or Gaussian) distribution:
\begin{align}
P(Z_k = s_i \mid X_k, \Delta t_k) &= P(\alpha_{m-1} < X_k'\beta+ \varepsilon_k \leq \alpha_1 \mid X_k, \Delta t_k) \\
&=
\begin{cases}
P(X_k' \beta + \varepsilon_k \leq \alpha_1 \mid X_k, \Delta t_k) & \text{if } i = 1, \\
P(\alpha_{i-1} < X_k' \beta + \varepsilon_k \leq \alpha_i \mid X_k, \Delta t_k) & \text{if } 1<i<m, \\
P(\alpha_{m-1} < X_k' \beta + \varepsilon_k \mid X_k, \Delta t_k) & \text{if } i = m,
\end{cases} \\
&=
\begin{cases}
\Phi\left( \frac{\alpha_1 - X_k' \beta}{\sigma_k} \right) & \text{if } i = 1, \\
\Phi\left( \frac{\alpha_i - X_k' \beta}{\sigma_k} \right) - \Phi\left( \frac{\alpha_{i-1} - X_k' \beta}{\sigma_k} \right) & \text{if } 1<i<m, \\
1 - \Phi\left( \frac{\alpha_{m-1} - X_k' \beta}{\sigma_k} \right) & \text{if } i=m,
\end{cases}
\label{eq:8}
\end{align}

where \(\Phi(\cdot)\) is the cumulative distribution function of standard normal distribution. Our model is then estimated with Maximum Likelihood via the following log-likelihood function with \verb|oglmx| package in R:

\begin{equation}
\label{eq:9}
\begin{split}
\mathcal{L}(Z \mid X)
&= \sum_{k=1}^n \Bigl\{
     Y_{1k}\,\log\Phi\!\Bigl(\tfrac{\alpha_1 - X_k'\beta}{\sigma_k}\Bigr) \\[6pt]
&\quad
   + \sum_{i=2}^{m-1} Y_{ik}\,\log\Bigl[
       \Phi\!\Bigl(\tfrac{\alpha_i - X_k'\beta}{\sigma_k}\Bigr)
     - \Phi\!\Bigl(\tfrac{\alpha_{i-1} - X_k'\beta}{\sigma_k}\Bigr)
     \Bigr] \\[6pt]
&\quad
   + Y_{mk}\,\log\Bigl[
       1 - \Phi\!\Bigl(\tfrac{\alpha_{m-1} - X_k'\beta}{\sigma_k}\Bigr)
     \Bigr]
 \Bigr\}\,.
\end{split}
\end{equation}





\section{Evaluation Metrics}