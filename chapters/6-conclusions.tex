\chapter{Conclusion}\label{chap:conclusion}
In this Chapter, the limitations and challenges imposing on this thesis are discussed. Possible future improvements are also explored. Finally, the Chapter concludes the main findings and remarks. 

\vspace{5mm}
{\noindent\bfseries Limitation, Challenges, \& Outlook }

This thesis faces several limiting factors. At first, the simplified ordered probit model does not incorporate the full list of explanatory variables proposed by \citet{hausman1992} that could impact the model's performance substantially, such as the S\&P 500 index that could be informative about the overall US financial market. The signed transformed volume is separatedly pre-transformed with natural log, while in the original study the Box-Cox transformation is estimated jointly with maximum likelihood estimation. Furthermore, the conditional variance that accounts for conditional heteroskedasticity is simplified to only time between trade. The restrictive usage of the \verb|oglmx| package constraints how the variance could be modeled. A customized, self-developed functions would be needed in the future to replicate the ordered probit model completely. Moreoever, due to time and resources constraint, so far this thesis only investigates IBM, a large and very liquid stock. To arrive at a robust and generalized conclusions, one should need more representative number and size of stocks. 

Secondly, the changing nature of the high frequency financial data characteristic, comparing to the 1992 period of the original paper, leaves this thesis further challenges in cleaning and pre-processing the data. One instant would be to verify whether our choice of trade direction classification's algorithm is already the most suitable choice or not, though the \% trade at price that is equal to mid-quote is minor. The issue of split transaction \citep{hautsch2012} is also not addressed in this thesis, which could be a major issue when not treated right, leading to the over-representation of the \textit{no change} price movement in the dataset.

Lastly, it would be more insightful to update the results with the more recent market microstructure theory regarding information-based trading. For the time being, the relationships between order flow and price change (or between trade size and price changes) have not been discussed at depth. The thesis simply revisits the test statistics and experiments from \citet{hausman1992} regarding the behavior of the market microstructure.




{\noindent\bfseries Concluding Remarks }

